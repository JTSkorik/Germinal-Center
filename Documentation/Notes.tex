\documentclass[english]{article}
\usepackage[T1]{fontenc}
\usepackage[latin9]{inputenc}
\usepackage{babel}
\usepackage{geometry}
\usepackage{hyperref}
\usepackage{amstext}
\usepackage{array}
\usepackage{tabularx}

\newcolumntype{C}[1]{>{\centering\arraybackslash}p{#1}}
\geometry{verbose,tmargin=1in,bmargin=1in,lmargin=1in,rmargin=1in}

\begin{document}

\title{Simulating a Germinal Center Notes}
\maketitle

\tableofcontents
\pagebreak	

\section{Resources.}
\textbf{Research Paper}: Robert, P., Rastogi, A., Binder, S. and Meyer-Hermann, M. (2017). How to Simulate a Germinal Center. Methods in Molecular Biology, pp.303-334.


\section{Model Assumptions.}
In this section, we highlight the assumptions made in the model we are following. This is primarily to refer to in future when making our own adjustments to the model. 
\begin{itemize}
\item Only one antigen per Germinal Center. 

\end{itemize}


\section{Implementation Ideas}

Generate $N\times N\times N$ spatial grid to place discrete sphere with radius $N/2$ within. Find and store the valid spatial points within this grid. At each time step, record where the discrete grid spot is free or contains a cell. Use dictionary, return None if empty, otherwise cell ID? If we want to find the position of a cell, we would need to work the dictionary backwards, likely to be computationally slow. Could have two dictionaries - one that gives position from cell ID and another that gives Cell ID / None from position. Is this likely to cause update issues?

The model given assigns cells an ID based on their position of each cell in its respective type list. This would allow for multiple cells to have the same ID and would require ID and type to be uniquely identified. Rather give each cell a unique ID regardless of type so always identifiable from ID alone. 

Don't need to have dictionary storing whether a point is within the sphere, can just re-apply the same calculation used to find original sphere co-ordinates. 

Let the points on the surface of the sphere be the points which are missing a neighbour. Should be easy to check.

Algorithms state to randomly iterate through list. Currently doing it by shuffling list and iterating over. Since elements sometimes have to be deleted from these lists, would be better to randomly shuffle a list of indices so the element known to be deleted can be removed straight away. Can make the removal operation quicker by popping end of list and placing it in position of removed element. 

Algorithm 4, progress\_cycle has been implemented poorly. Instead of having a lot of if statements, could store the states in a list in order of how they transition. Then only have to store the index for each cell and when they progress, add one to that index opposed to using lots of if statements. 

The following was followed to generate random unit vectors for initial polarity: https://codereview.stackexchange.com/questions/77927/generate-random-unit-vectors-around-circle

The germinal center has a maximum size. We generate ID values from 0 to the maximum number of cells in the germinal center and store/track them in a list. We assign each new cell an ID number (from the end of the list, using pop() to ensure O(1) operation) and when a cell dies, we append its ID value back into the list. In a numpy array of fixed size, we will store the properties of each cell as a named tuple (Not using named tuples since we can not charge their properties, instead, we will use types.SimpleNamespace). To access the correct properties, the associated ID to a cell will be used as the index in the numpy array (Now using lists since numpy arrays can't store SimpleNamespace variables). Unassigned IDs will contain None values in numpy array (list) and when a cell gets removed from the simulation, the numpy array (list) index is converted back to being a None value. 

In main function, we have to iterate over lists randomly and occasionally remove elements from this list, can do so using enumerate function to track indices of what needs to be removed. 

Can pass a cell object to a new function, edit it and not have to return the cell object for changes to be reflected, they'll occur automatically. 



\section{Parameters for each Cell}

\subsection{Stromal cell}
\begin{center}
\begin{tabular}{|C{2.1in}|C{2.1in}|C{2.1in}|}
\hline
\textbf{Property} & \textbf{Data Type} & \textbf{Description}
\tabularnewline
\hline
\hline
Type & Enumeration & The type of cell. 
\tabularnewline
\hline
Position & Tuple & Position within GC.
\tabularnewline
\hline

\end{tabular}
\end{center}

\subsection{F-Cell}
\begin{center}
\begin{tabular}{|C{2.1in}|C{2.1in}|C{2.1in}|}

\hline
\textbf{Property} & \textbf{Data Type} & \textbf{Description}
\tabularnewline
\hline
\hline
Type & Enumeration & The type of cell. 
\tabularnewline
\hline
Position & Tuple & Position within GC.
\tabularnewline
\hline
antigenAmount & Float & Amount of Antigen Retained by the Fragment. 
\tabularnewline
\hline
icAmount & Float &
\tabularnewline
\hline
Fragments & List of IDs(int) & List of IDs for each fragment of given F-cell. 
\tabularnewline
\hline

\end{tabular}
\end{center}

\subsection{Fragment}
\begin{center}
\begin{tabular}{|C{2.1in}|C{2.1in}|C{2.1in}|}

\hline
\textbf{Property} & \textbf{Data Type} & \textbf{Description}
\tabularnewline
\hline
\hline
Type & Enumeration & The type of cell. 
\tabularnewline
\hline
Position & Tuple & Position within GC.
\tabularnewline
\hline
antigenAmount & Float & Amount of Antigen Retained by the Fragment. 
\tabularnewline
\hline
icAmount & Float &
\tabularnewline
\hline 
Parent & Integer & ID for center of F cell.  
\tabularnewline
\hline

\end{tabular}
\end{center}

\subsection{Centroblast}

\begin{center}
\begin{tabular}{|C{2.1in}|C{2.1in}|C{2.1in}|}

\hline
\textbf{Property} & \textbf{Data Type} & \textbf{Description}
\tabularnewline
\hline
\hline
Type & Enumeration & The type of cell. 
\tabularnewline
\hline
Position & Tuple & Position within GC.
\tabularnewline
\hline
State & Enumeration & Current state of the cell. 
\tabularnewline
\hline
BCR & 4 Digit integer & BCR value for cell.
\tabularnewline
\hline
Polarity & 3D Numpy Array / Vector & Polarity of cell.
\tabularnewline
\hline
responsiveToCXCL12 & Boolean & Records whether cell is responsive to signal CXCL12.
\tabularnewline
\hline
responsiveToCXCL13 & Boolean & Records whether cell is responsive to signal CXCL13.
\tabularnewline
\hline
numDivisionsToDo & Integer & The number of divisions the cell is yet to do. 
\tabularnewline
\hline 
pMutation & Float & Probability of the cell mutating. 
\tabularnewline
\hline
IAmHighAg & Boolean & 
\tabularnewline
\hline
retainedAg & Float & Amount of antigen retained by the cell. 
\tabularnewline
\hline
cycleStartTime & Float & Amount of time spent in current state. 
\tabularnewline
\hline
endOfThisPhase & Float & Time at which the cell will finish being in this state. 
\tabularnewline
\hline

\end{tabular}
\end{center}


\subsection{Centrocyte}

\begin{center}
\begin{tabular}{|C{2.1in}|C{2.1in}|C{2.1in}|}

\hline
\textbf{Property} & \textbf{Data Type} & \textbf{Description}
\tabularnewline
\hline
\hline
Type & Enumeration & The type of cell. 
\tabularnewline
\hline
Position & Tuple & Position within GC.
\tabularnewline
\hline
State & Enumeration & Current state of the cell. 
\tabularnewline
\hline
BCR & 4 Digit integer & BCR value for cell.
\tabularnewline
\hline
Polarity & 3D Numpy Array / Vector & Polarity of cell.
\tabularnewline
\hline
responsiveToCXCL12 & Boolean & Records whether cell is responsive to signal CXCL12.
\tabularnewline
\hline
responsiveToCXCL13 & Boolean & Records whether cell is responsive to signal CXCL13.
\tabularnewline
\hline
selectedClock & Float &
\tabularnewline
\hline
Clock & Float &
\tabularnewline
\hline
selectable & Boolean &
\tabularnewline
\hline
FragContact & None or Integer & 
\tabularnewline
\hline
numFDCcontacts & Integer &
\tabularnewline
\hline
tcClock & Float &
\tabularnewline
\hline
tcSignalDuration & Float &
\tabularnewline
\hline
individualDifDelay & Float &
\tabularnewline
\hline
TCell\_Contact & None or Integer & If in contact with T cell, this stores the ID of said T cell. 
\tabularnewline
\hline
\end{tabular}
\end{center}

\subsection{T cell}
\begin{center}
\begin{tabular}{|C{2.1in}|C{2.1in}|C{2.1in}|}
\hline
\textbf{Property} & \textbf{Data Type} & \textbf{Description}
\tabularnewline
\hline
\hline
Type & Enumeration & The type of cell. 
\tabularnewline
\hline
Position & Tuple & Position within GC.
\tabularnewline
\hline
State & Enumeration & Current state of the cell.
\tabularnewline
\hline
Polarity & 3D Numpy Array / Vector & Polarity of cell.
\tabularnewline
\hline
BCell\_Contacts & List of integers & List of IDs of B cells (Centrocytes) in contact with T cell. 
\tabularnewline
\hline

\end{tabular}
\end{center}

\subsection{Outcell}
\begin{center}
\begin{tabular}{|C{2.1in}|C{2.1in}|C{2.1in}|}
\hline
\textbf{Property} & \textbf{Data Type} & \textbf{Description}
\tabularnewline
\hline
\hline
Type & Enumeration & The type of cell. 
\tabularnewline
\hline
Position & Tuple & Position within GC.
\tabularnewline
\hline
Polarity & 3D Numpy Array / Vector & Polarity of cell.
\tabularnewline
\hline
responsiveToCXCL12 & Boolean & Records whether cell is responsive to signal CXCL12.
\tabularnewline
\hline
responsiveToCXCL13 & Boolean & Records whether cell is responsive to signal CXCL13.
\tabularnewline
\hline

\end{tabular}
\end{center}


\subsection{General Parameters \& Tracked Lists}
\begin{itemize}

\item We keep two dictionaries, Grid\_id and Grid\_type. These dictionaries have a key of a tuple location within the sphere and return the cell and cell type located at that location, respectively. If that position is free, both dictionaries will contain 'None'.

\item Dictionaries CXCL12 and CXCL12 will stores the amount of each located at a given location. 

\item We will use StormaList, FDCList, CBList, and TCList to store the IDs of cells in each of these respective states. 

\end{itemize}

\section{Definitions \& Terms.}

\begin{itemize}


\item \textbf{Affinity Maturation}: The process by which Tfh cell-activated B cells produce antibodies with increase affinity for antigen during the course of an immune response. 

\item \textbf{Stromal Cells}: Connective tissue cells of any organ.
 
\item \textbf{B Cells}: Type of white blood cell. 

\item \textbf{Centroblasts}: B cell that is enlarged and proliferating in the germinal center.

\item \textbf{Clonal Expansion}: A large increase in the number of B cells and T cells in the presence of an infection. 

\item \textbf{Somatic Hypermutation (SHM)}: A cellular mechanism by which the immune system adapts to the new foreign elements that confront it. Allows B cells to mutate the genes that they use to produce antibodies. 

\item \textbf{B Cell Receptor (BCR)}:

\item \textbf{VDJ Recombination Pattern}: 

\item \textbf{in vivo}: A study where the effects of various biological entities are tested on whole, living organisms or cells. 

\item \textbf{Centrocytes}: Nondividing B cells that endure a high apoptosis rate. 

\item \textbf{Follicular Dendritic Cells (FDCs)}: A type of cell in the immune system.

\item \textbf{Fc Receptors}: A protein found on the surface of certain cells that contributes to the protective functions of the immune system. 

\item \textbf{MHC Class II}: A class of molecules normally found only on antigen presenting cells. Important in initiating immune responses. 

\item \textbf{Antigen Presenting Cell (APC)}: A cell that displays antigen complexed with major histocompatibility complexes (MHCs) on their surfaces.

\item \textbf{T Helper Cells}: Cells that help the activity of other immune cells by releasing T cell cytokines (small proteins).

\item \textbf{T Follicular Helper (Tfh)}: Within a Germinal Center they mediate the selection and survival of B cells that differentiate either into special plasma cells capable of producing high affinity antibodies against foreign antigen, or memory B cells capable of quick immune re-activation in the future if the same antigen is ever accounted again. 


\end{itemize}

\section{Misc.}
\subsection{Questions.}
\begin{itemize}

\item The description at the start shows that the cells stay in either the light or the dark zone. In the simulations, we define these as the lower and upper parts of the sphere.  Centroblasts can eventually become centrocytes, which reside in different parts. The algorithms don't force the cell to be in the right zone after converting. 


\end{itemize}

\subsection{To do list}

\begin{itemize}

\item Class parameters

\item dictionaries

\item replace grid dictionaries with numpy array

\item watch video and read papers

\item look at docstring/html

\end{itemize}

\end{document}