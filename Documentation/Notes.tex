\documentclass[english]{article}
\usepackage[T1]{fontenc}
\usepackage[latin9]{inputenc}
\usepackage{babel}
\usepackage{geometry}
\usepackage{hyperref}
\usepackage{amstext}
\usepackage{array}
\usepackage{tabularx}
\usepackage{bm}

\newcolumntype{C}[1]{>{\centering\arraybackslash}p{#1}}
\geometry{verbose,tmargin=1in,bmargin=1in,lmargin=1in,rmargin=1in}

\begin{document}

\title{Simulating a Germinal Center Notes}
\maketitle

\tableofcontents
\pagebreak	

\section{Resources.}
\textbf{Research Paper}: Robert, P., Rastogi, A., Binder, S. and Meyer-Hermann, M. (2017). How to Simulate a Germinal Center. Methods in Molecular Biology, pp.303-334.


\section{Implementation Notes}
\begin{itemize}


\item Note that the size of the GC is constant. This gives us an upper bound on the number of cells. We generate a list of integers with as many unique entries as the size of the GC. When a new cell is created, we assign it an ID from this list. When a cell leaves the simulation, we add its ID back into the list. We use this unique ID as a key for each cells properties in respective cell types dictionaries. 

\item Since the sphere is shifted to be in the first octant, we can create a 3D numpy array where each element corresponds to a position in the GC. We set points outside the GC to -1 and empty points within the GC to None. In the filled positions we store the ID of the cell in that position. This allows for O(1) checks for where a position is free or inside the GC. 

\item We set the surface layer points of the GC to be those that have a neighbour outside of the GC. 

\item The following was followed to generate random unit vectors for initial polarity: https://codereview.stackexchange.com/questions/77927/generate-random-unit-vectors-around-circle
Summary: Can sample three $\mathcal{N}(0,1)$ variables and normalise to obtain a vector on unit sphere. 

\item Rotating Polarity: Instructions were ambiguous, here is the implemented interpretation. We find a vector $\bm{v}$ that is perpendicular to the polarity vector. We first rotate the polarity $\theta$ degrees towards the vector $\bm{v}$. $\theta$ is sampled from a distribution not given, replaced with Normal distribution. Next, we rotate the resulting vector by $\phi$ degrees around the original polarity vector. Here, $\phi$ is sampled from uniform$(0,2\pi)$. Since $\bm{v}$ is chosen randomly and $\phi$ is chosen uniformly at random, these actions are equivalent to rotating the original polarity vector by $\theta$ degrees around a randomly generated perpendicular vector $\bm{v}$. We can generate $\bm{v}$ using
\begin{equation}
\bm{v}=\bm{r} - (\bm{r}\cdot\bm{n})\bm{n},
\end{equation}
where $\bm{r}$ is a random unit vector, and $\bm{n}$ is the polarity vector. We can then use the rotation matrix
\begin{equation}
\left[\begin{array}{ccc}
\cos(\theta)+v_{x}^{2}\left(1-\cos(\theta)\right) & v_{x}v_{y}\left(1-\cos(\theta)\right)-v_{z}\sin(\theta) & v_{x}v_{z}\left(1-\cos(\theta)\right)+v_{y}\sin(\theta)\\
v_{y}v_{x}\left(1-\cos(\theta)\right)+v_{z}\sin(\theta) & \cos(\theta)+v_{y}^{2}\left(1-\cos(\theta)\right) & v_{y}v_{z}\left(1-\cos(\theta)\right)-v_{x}\sin(\theta)\\
v_{z}v_{x}\left(1-\cos(\theta)\right)-v_{y}\sin(\theta) & v_{z}v_{y}\left(1-\cos(\theta)\right)+v_{x}\sin(\theta) & \cos(\theta)+v_{z}^{2}\left(1-\cos(\theta)\right)
\end{array}\right],
\end{equation}
to rotate the polarity around $\bm{v}=(v_x, v_y, v_z)$.

\item Diffusion of CXCL12 and CXCL13:


\item Secrete CXCL12 and CXCL13:




	
 




\end{itemize}

\newpage
\section{Parameters for each Cell}

\subsection{Stromal cell}
\begin{center}
\begin{tabular}{|C{2.1in}|C{2.1in}|C{2.1in}|}
\hline
\textbf{Property} & \textbf{Data Type} & \textbf{Description}
\tabularnewline
\hline
\hline
Type & Enumeration & The type of cell. 
\tabularnewline
\hline
Position & Tuple & Position within GC.
\tabularnewline
\hline

\end{tabular}
\end{center}

\subsection{F-Cell}
\begin{center}
\begin{tabular}{|C{2.1in}|C{2.1in}|C{2.1in}|}

\hline
\textbf{Property} & \textbf{Data Type} & \textbf{Description}
\tabularnewline
\hline
\hline
Type & Enumeration & The type of cell. 
\tabularnewline
\hline
Position & Tuple & Position within GC.
\tabularnewline
\hline
antigenAmount & Float & Amount of Antigen Retained by the Fragment. 
\tabularnewline
\hline
icAmount & Float &
\tabularnewline
\hline
Fragments & List of IDs(int) & List of IDs for each fragment of given F-cell. 
\tabularnewline
\hline

\end{tabular}
\end{center}

\subsection{Fragment}
\begin{center}
\begin{tabular}{|C{2.1in}|C{2.1in}|C{2.1in}|}

\hline
\textbf{Property} & \textbf{Data Type} & \textbf{Description}
\tabularnewline
\hline
\hline
Type & Enumeration & The type of cell. 
\tabularnewline
\hline
Position & Tuple & Position within GC.
\tabularnewline
\hline
antigenAmount & Float & Amount of Antigen Retained by the Fragment. 
\tabularnewline
\hline
icAmount & Float &
\tabularnewline
\hline 
Parent & Integer & ID for center of F cell.  
\tabularnewline
\hline

\end{tabular}
\end{center}

\subsection{Centroblast}

\begin{center}
\begin{tabular}{|C{2.1in}|C{2.1in}|C{2.1in}|}

\hline
\textbf{Property} & \textbf{Data Type} & \textbf{Description}
\tabularnewline
\hline
\hline
Type & Enumeration & The type of cell. 
\tabularnewline
\hline
Position & Tuple & Position within GC.
\tabularnewline
\hline
State & Enumeration & Current state of the cell. 
\tabularnewline
\hline
BCR & 4 Digit integer & BCR value for cell.
\tabularnewline
\hline
Polarity & 3D Numpy Array / Vector & Polarity of cell.
\tabularnewline
\hline
responsiveToCXCL12 & Boolean & Records whether cell is responsive to signal CXCL12.
\tabularnewline
\hline
responsiveToCXCL13 & Boolean & Records whether cell is responsive to signal CXCL13.
\tabularnewline
\hline
numDivisionsToDo & Integer & The number of divisions the cell is yet to do. 
\tabularnewline
\hline 
pMutation & Float & Probability of the cell mutating. 
\tabularnewline
\hline
IAmHighAg & Boolean & 
\tabularnewline
\hline
retainedAg & Float & Amount of antigen retained by the cell. 
\tabularnewline
\hline
cycleStartTime & Float & Amount of time spent in current state. 
\tabularnewline
\hline
endOfThisPhase & Float & Time at which the cell will finish being in this state. 
\tabularnewline
\hline

\end{tabular}
\end{center}


\subsection{Centrocyte}

\begin{center}
\begin{tabular}{|C{2.1in}|C{2.1in}|C{2.1in}|}

\hline
\textbf{Property} & \textbf{Data Type} & \textbf{Description}
\tabularnewline
\hline
\hline
Type & Enumeration & The type of cell. 
\tabularnewline
\hline
Position & Tuple & Position within GC.
\tabularnewline
\hline
State & Enumeration & Current state of the cell. 
\tabularnewline
\hline
BCR & 4 Digit integer & BCR value for cell.
\tabularnewline
\hline
Polarity & 3D Numpy Array / Vector & Polarity of cell.
\tabularnewline
\hline
responsiveToCXCL12 & Boolean & Records whether cell is responsive to signal CXCL12.
\tabularnewline
\hline
responsiveToCXCL13 & Boolean & Records whether cell is responsive to signal CXCL13.
\tabularnewline
\hline
selectedClock & Float &
\tabularnewline
\hline
Clock & Float &
\tabularnewline
\hline
selectable & Boolean &
\tabularnewline
\hline
FragContact & None or Integer & 
\tabularnewline
\hline
numFDCcontacts & Integer &
\tabularnewline
\hline
tcClock & Float &
\tabularnewline
\hline
tcSignalDuration & Float &
\tabularnewline
\hline
individualDifDelay & Float &
\tabularnewline
\hline
TCell\_Contact & None or Integer & If in contact with T cell, this stores the ID of said T cell. 
\tabularnewline
\hline
\end{tabular}
\end{center}

\subsection{T cell}
\begin{center}
\begin{tabular}{|C{2.1in}|C{2.1in}|C{2.1in}|}
\hline
\textbf{Property} & \textbf{Data Type} & \textbf{Description}
\tabularnewline
\hline
\hline
Type & Enumeration & The type of cell. 
\tabularnewline
\hline
Position & Tuple & Position within GC.
\tabularnewline
\hline
State & Enumeration & Current state of the cell.
\tabularnewline
\hline
Polarity & 3D Numpy Array / Vector & Polarity of cell.
\tabularnewline
\hline
BCell\_Contacts & List of integers & List of IDs of B cells (Centrocytes) in contact with T cell. 
\tabularnewline
\hline

\end{tabular}
\end{center}

\subsection{Outcell}
\begin{center}
\begin{tabular}{|C{2.1in}|C{2.1in}|C{2.1in}|}
\hline
\textbf{Property} & \textbf{Data Type} & \textbf{Description}
\tabularnewline
\hline
\hline
Type & Enumeration & The type of cell. 
\tabularnewline
\hline
Position & Tuple & Position within GC.
\tabularnewline
\hline
Polarity & 3D Numpy Array / Vector & Polarity of cell.
\tabularnewline
\hline
responsiveToCXCL12 & Boolean & Records whether cell is responsive to signal CXCL12.
\tabularnewline
\hline
responsiveToCXCL13 & Boolean & Records whether cell is responsive to signal CXCL13.
\tabularnewline
\hline

\end{tabular}
\end{center}


\newpage
\section{Definitions \& Terms.}

\begin{itemize}


\item \textbf{Affinity Maturation}: The process by which Tfh cell-activated B cells produce antibodies with increase affinity for antigen during the course of an immune response. 

\item \textbf{Stromal Cells}: Connective tissue cells of any organ.
 
\item \textbf{B Cells}: Type of white blood cell. 

\item \textbf{Centroblasts}: B cell that is enlarged and proliferating in the germinal center.

\item \textbf{Clonal Expansion}: A large increase in the number of B cells and T cells in the presence of an infection. 

\item \textbf{Somatic Hypermutation (SHM)}: A cellular mechanism by which the immune system adapts to the new foreign elements that confront it. Allows B cells to mutate the genes that they use to produce antibodies. 

\item \textbf{B Cell Receptor (BCR)}:

\item \textbf{VDJ Recombination Pattern}: 

\item \textbf{in vivo}: A study where the effects of various biological entities are tested on whole, living organisms or cells. 

\item \textbf{Centrocytes}: Nondividing B cells that endure a high apoptosis rate. 

\item \textbf{Follicular Dendritic Cells (FDCs)}: A type of cell in the immune system.

\item \textbf{Fc Receptors}: A protein found on the surface of certain cells that contributes to the protective functions of the immune system. 

\item \textbf{MHC Class II}: A class of molecules normally found only on antigen presenting cells. Important in initiating immune responses. 

\item \textbf{Antigen Presenting Cell (APC)}: A cell that displays antigen complexed with major histocompatibility complexes (MHCs) on their surfaces.

\item \textbf{T Helper Cells}: Cells that help the activity of other immune cells by releasing T cell cytokines (small proteins).

\item \textbf{T Follicular Helper (Tfh)}: Within a Germinal Center they mediate the selection and survival of B cells that differentiate either into special plasma cells capable of producing high affinity antibodies against foreign antigen, or memory B cells capable of quick immune re-activation in the future if the same antigen is ever accounted again. 


\end{itemize}


\end{document}