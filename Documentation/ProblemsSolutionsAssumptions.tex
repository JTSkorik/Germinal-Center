
\documentclass[english]{article}
\usepackage[T1]{fontenc}
\usepackage[latin9]{inputenc}
\usepackage{geometry}
\geometry{verbose,tmargin=1in,bmargin=1in,lmargin=1in,rmargin=1in}
\usepackage{array}
\usepackage{longtable}

\makeatletter


\makeatother

\usepackage{babel}
\begin{document}
\begin{center}
\begin{longtable}{|>{\centering}p{2.1in}|>{\centering}p{2.1in}|>{\centering}p{2.1in}|}
\hline 
Problem & Solution(s) & Assumption(s)\tabularnewline
\hline 
\hline 
When applying mutation to BCR (4 Digit Code), it is not clear how
to decide where to plus or minus one. & If both are possible, apply either with equal chance of occurring.
If only one is possible, apply that change.  & Both options equally likely. \tabularnewline
\hline 
We require a probability distribution to determine the turning angle.
Distribution not found in the cited references. & Can estimate it using a Random Variable.   & Turning angle is equally likely to go ``left'' or ``right''\tabularnewline
\hline 
In algorithm 3: ``with probability persistentLengthTime(C.type)''.
In parameters, all values but one of these parameters have a value
less than 1.  & Value is given in terms of minutes. We will convert this to minutes
so that all cases have a probability less than one. Worth noting this
probability will be tiny after the conversion.  & -\tabularnewline
\hline 
In algorithm 9: ``get new BCR, randomly or from a predefined repertoire''.  & Uniformally generate BCR values for now.  & -\tabularnewline
\hline 
Require amount of CXCL12 and CXCL13 at each position within the germinal
center.  & Use Guassian or Uniform random variable to assign temporary values. & Assumed values are distributed similar to critical values given in
table of values\tabularnewline
\hline 
 & Assign value near critical value to all positions in cell.  & Assumed the germinal center starts off with a constant amount throughout. \tabularnewline
\hline 
pDif2Out(time) has a formula with a variable that's always zero. & - & -\tabularnewline
\hline 
In algorithm 9 it is not clear whether the FDC branches remain within
the Light Zone.  & Forced branches to be only within light zone.  & As suggested by diagrams at the start of the paper, the FDC branches
cannot leave the Light Zone. \tabularnewline
\hline 
Algorithm 10 contains two function that have not been described.  &  & \tabularnewline
\hline 
The initial polarity of each cell is unknown.  & Randomly assign the polarity. & \tabularnewline
\hline 

Algorithm 3. The move function has a vector called North. North is
a vector pointing towards the light zone and is not defined. This
is peculiar as it only influences T cells and they are already in
the Light Zone.  & Light zone is the bottom of the sphere, could use $(0,0,-1)$. & \tabularnewline
\hline 

Algorithm 3, in the move function we must not move the cell against the polarity. Hard to determine this. & We find the best possible points the cell can move to, in order and only consider the first 8 points, taking the best possible. If none of these points are free, the cell does not move.& 
\tabularnewline
\hline 

Not explicitly stated if a cell is able to divide and have new cell a diagonal movement away. Similar for if we want to know if one type of Cell is next to another. & Allow cells to divide into positions diagonally away.  & 
\tabularnewline
\hline

It is not clear how to determine where a cell is IAmHighAg and it does not state where a cell starts off as True or False. & Initially start all cells if IAmHighAg set to False. &
\tabularnewline
\hline

It is not specified whether the 4 digit BCR value can start with a zero. Algorithm 1 suggests it can while figure 4 suggests it cannot. & We will not let the value start with zero. Python will automatically convert it to a 3 digit number and that will cause issues.  & Assume it cannot start with a zero. 
\tabularnewline
\hline

Algorithm 4, sampling Gaussian Distribution to find a value for sep, the mean is given but standard deviation is not. & Let standard deviation be one. &
\tabularnewline
\hline

Algorithm 5, C.clock suggest we add one to the clock counter. Next line says to test if C.clock > testDelay, which will always be true since testDelay is 0.02 . & Changed to C.clock increment by dt. & Assumed algorithm has a type. From notes, can somewhat assume C.clock is supposed to be C.tcClock which increments by dt and is not mentioned in the algorithms. 
\tabularnewline
\hline

Algorithm 5, lines 8 and 9. What if he have multiple fragments next to the cell? & Found the maximum expression for the formula given on line 9. & Can only bond with one and the antigen amounts from multiple neighbouring fragments would not add. Bonds with the one with most	 antigen. 
\tabularnewline
\hline

Algorithm 5 has two main if statements. The first if statement can change the state of a cell such that it also triggers the second if state. Same problem in algorithm 6. & Using an If and else if statement. & Assumed that only one if statement should be carried out since they both increment a time variable by the time step, dt. Would not make much sense having that increment twice in one time step. 
\tabularnewline
\hline

Algorithm 5, line 31: AntigenAmount[f]- -, Antigen amount seems to be a float but this code says to remove one from it, seems strange. & Follow what it says exactly, subtract one from it. & Assume that a cell can take a certain amount of antigen from a fragment and without loss of generality, we can set this quantity to be one. 
\tabularnewline
\hline 

Algorithm 6, line 13. Refers to B cells having antigens. & Treat this as amount of retained antigen. & 
\tabularnewline
\hline

Algorithm 6, line 13 also does not say what to do if a B cell is in contact with multiple T cells. & We will only consider the first T cell in contact with a B cell in the first main if statement of the algorithm. This is not a great solution and will probably need refining.  & 
\tabularnewline
\hline

When calculating affinity, we need to find hamming distance between B cell BCR and Antigen. Diagram suggests that we use something slightly different to the hamming distance. & We will use the traditional hamming distance. &
\tabularnewline
\hline

Not given BCR values or antigen value. & Randomly generate 1000 four-digit values to act as BCR values. Let the antigen value be 1234 as a placeholder. &
\tabularnewline
\hline

Algorithm 10, lines 18, 19, 20. We calculate transfert then subtract it from NumBCROutCells and NumBCROutCellsProduce. This calculation is almost surely going to end in a float. & Floor the result to return an integer. & 
\tabularnewline
\hline

Move function has no specifications forcing certain types of cells in the light or dark zone.  & Will include a check that ensures certain types of cells remain within their respective zones. &
\tabularnewline
\hline

Algorithm 10, line 10 refers to a variable $d_{t}$, referenced elsewhere. & Let it be dt. & Assume type was made. 
\tabularnewline
\hline

Does not state whether T cells are responsive to CXCL12 and CXCL13, which is required for move function. & Set both to False & Assume unresponsive. 
\tabularnewline
\hline

\end{longtable}
\par\end{center}
\end{document}