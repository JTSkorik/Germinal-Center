
\documentclass[english]{article}
\usepackage[T1]{fontenc}
\usepackage[latin9]{inputenc}
\usepackage{geometry}
\geometry{verbose,tmargin=1in,bmargin=1in,lmargin=1in,rmargin=1in}
\usepackage{array}
\usepackage{longtable}

\makeatletter


\makeatother

\usepackage{babel}
\begin{document}
\begin{center}
\begin{longtable}{|>{\centering}p{2.1in}|>{\centering}p{2.1in}|>{\centering}p{2.1in}|}
\hline 
Problem & Solution(s) & Assumption(s)\tabularnewline
\hline 
\hline 
When applying mutation to BCR (4 Digit Code), it is not clear how
to decide where to plus or minus one. & If both are possible, apply either with equal chance of occurring.
If only one is possible, apply that change.  & Both options equally likely. \tabularnewline
\hline 
We require a probability distribution to determine the turning angle.
Distribution not found in the cited references. & Can estimate it using a Random Variable.   & Turning angle is equally likely to go ``left'' or ``right''\tabularnewline
\hline 
In algorithm 3: ``with probability persistentLengthTime(C.type)''.
In parameters, all values but one of these parameters have a value
less than 1.  & Value is given in terms of minutes. We will convert this to minutes
so that all cases have a probability less than one. Worth noting this
probability will be tiny after the conversion.  & -\tabularnewline
\hline 
In algorithm 9: ``get new BCR, randomly or from a predefined repertoire''.  & Uniformally generate BCR values for now.  & -\tabularnewline
\hline 
Require amount of CXCL12 and CXCL13 at each position within the germinal
center.  & Use Guassian or Uniform random variable to assign temporary values. & Assumed values are distributed similar to critical values given in
table of values\tabularnewline
\hline 
 & Assign value near critical value to all positions in cell.  & Assumed the germinal center starts off with a constant amount throughout. \tabularnewline
\hline 
pDif2Out(time) has a formula with a variable that's always zero. & - & -\tabularnewline
\hline 
In algorithm 9 it is not clear whether the FDC branches remain within
the Light Zone.  & Forced branches to be only within light zone.  & As suggested by diagrams at the start of the paper, the FDC branches
cannot leave the Light Zone. \tabularnewline
\hline 
Algorithm 10 contains two function that have not been described.  &  & \tabularnewline
\hline 
The initial polarity of each cell is unknown.  & Randomly assign the polarity. & \tabularnewline
\hline 
Algorithm 3. The move function has a vector called North. North is
a vector pointing towards the light zone and is not defined. This
is peculiar as it only influences T cells and they are already in
the Light Zone.  & Light zone is the bottom of the sphere, could use $(0,0,-1)$. & \tabularnewline
\hline 
Algorithm 3, in the move function we must not move the cell against the polarity. Hard to determine this. & We find the best possible points the cell can move to, in order and only consider the first 8 points, taking the best possible. If none of these points are free, the cell does not move.& 
\tabularnewline
\hline 

\end{longtable}
\par\end{center}
\end{document}